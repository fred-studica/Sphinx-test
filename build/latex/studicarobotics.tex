%% Generated by Sphinx.
\def\sphinxdocclass{report}
\documentclass[letterpaper,10pt,english]{sphinxmanual}
\ifdefined\pdfpxdimen
   \let\sphinxpxdimen\pdfpxdimen\else\newdimen\sphinxpxdimen
\fi \sphinxpxdimen=.75bp\relax

\PassOptionsToPackage{warn}{textcomp}
\usepackage[utf8]{inputenc}
\ifdefined\DeclareUnicodeCharacter
% support both utf8 and utf8x syntaxes
  \ifdefined\DeclareUnicodeCharacterAsOptional
    \def\sphinxDUC#1{\DeclareUnicodeCharacter{"#1}}
  \else
    \let\sphinxDUC\DeclareUnicodeCharacter
  \fi
  \sphinxDUC{00A0}{\nobreakspace}
  \sphinxDUC{2500}{\sphinxunichar{2500}}
  \sphinxDUC{2502}{\sphinxunichar{2502}}
  \sphinxDUC{2514}{\sphinxunichar{2514}}
  \sphinxDUC{251C}{\sphinxunichar{251C}}
  \sphinxDUC{2572}{\textbackslash}
\fi
\usepackage{cmap}
\usepackage[T1]{fontenc}
\usepackage{amsmath,amssymb,amstext}
\usepackage{babel}



\usepackage{times}
\expandafter\ifx\csname T@LGR\endcsname\relax
\else
% LGR was declared as font encoding
  \substitutefont{LGR}{\rmdefault}{cmr}
  \substitutefont{LGR}{\sfdefault}{cmss}
  \substitutefont{LGR}{\ttdefault}{cmtt}
\fi
\expandafter\ifx\csname T@X2\endcsname\relax
  \expandafter\ifx\csname T@T2A\endcsname\relax
  \else
  % T2A was declared as font encoding
    \substitutefont{T2A}{\rmdefault}{cmr}
    \substitutefont{T2A}{\sfdefault}{cmss}
    \substitutefont{T2A}{\ttdefault}{cmtt}
  \fi
\else
% X2 was declared as font encoding
  \substitutefont{X2}{\rmdefault}{cmr}
  \substitutefont{X2}{\sfdefault}{cmss}
  \substitutefont{X2}{\ttdefault}{cmtt}
\fi


\usepackage[Bjarne]{fncychap}
\usepackage{sphinx}

\fvset{fontsize=\small}
\usepackage{geometry}

% Include hyperref last.
\usepackage{hyperref}
% Fix anchor placement for figures with captions.
\usepackage{hypcap}% it must be loaded after hyperref.
% Set up styles of URL: it should be placed after hyperref.
\urlstyle{same}
\addto\captionsenglish{\renewcommand{\contentsname}{World Skills}}

\usepackage{sphinxmessages}
\setcounter{tocdepth}{0}



\title{Studica Robotics}
\date{Apr 29, 2020}
\release{0.0.1}
\author{Fred}
\newcommand{\sphinxlogo}{\vbox{}}
\renewcommand{\releasename}{Release}
\makeindex
\begin{document}

\pagestyle{empty}
\sphinxmaketitle
\pagestyle{plain}
\sphinxtableofcontents
\pagestyle{normal}
\phantomsection\label{\detokenize{index::doc}}



\chapter{World Skills Software Setup Guide}
\label{\detokenize{docs/WorldSkills/setup:world-skills-software-setup-guide}}\label{\detokenize{docs/WorldSkills/setup::doc}}
\begin{sphinxadmonition}{note}{Note:}
This setup is for Java/C++ only
\end{sphinxadmonition}


\section{Online Installer}
\label{\detokenize{docs/WorldSkills/setup:online-installer}}
\begin{sphinxadmonition}{caution}{Caution:}
The online installation will download approximately 1.3GB worth of data, please maintain a stable conneciton while downloading
\end{sphinxadmonition}

Windows

macOS

Linux

\begin{sphinxadmonition}{warning}{Warning:}
Windows 7: You \sphinxstylestrong{must} install the standalone version of .NET Version 4.62+ which can be found \sphinxhref{https://support.microsoft.com/en-us/help/3151800/the-net-framework-4-6-2-offline-installer-for-windows}{here}. Before preceding!
\end{sphinxadmonition}

Download the installer (32bit or 64bit) \sphinxhref{https://github.com/wpilibsuite/allwpilib/releases}{from WPI}.

\noindent{\hspace*{\fill}\sphinxincludegraphics{{worldskills-software-setup-1}.PNG}\hspace*{\fill}}

Installing for \sphinxstyleemphasis{All Users} will require admin privileges and install for all users on the machine.

\begin{sphinxadmonition}{note}{Note:}
Software will be installed to \sphinxcode{\sphinxupquote{C:\textbackslash{}Users\textbackslash{}Public\textbackslash{}wpilib\textbackslash{}YYYY}}. YYYY Corresponds to the currently suppored year.
\end{sphinxadmonition}

\sphinxstylestrong{Download VS Code}
\begin{quote}

Due to licensing reasons with VS Code the installer does not contain it bundled in. To overcome this hit the \sphinxstyleemphasis{Select/Download VS Code} button.

\noindent{\hspace*{\fill}\sphinxincludegraphics{{worldskills-software-setup-2}.PNG}\hspace*{\fill}}

This will open up the selector tool.

\noindent{\hspace*{\fill}\sphinxincludegraphics{{worldskills-software-setup-3}.PNG}\hspace*{\fill}}

Select the \sphinxstyleemphasis{Download} option and VS Code will be downloaded.

\noindent{\hspace*{\fill}\sphinxincludegraphics{{worldskills-software-setup-4}.PNG}\hspace*{\fill}}

Once VS Code is downloaded the window will auto change back to the installer window and \sphinxstyleemphasis{Execute Install} can be run.

\noindent{\hspace*{\fill}\sphinxincludegraphics{{worldskills-software-setup-5}.PNG}\hspace*{\fill}}
\end{quote}

\sphinxstylestrong{What was just Installed}
\begin{itemize}
\item {} 
\sphinxstyleemphasis{Visual Stuido Code} - The prefered and suppored IDE for robot code development.

\item {} 
\sphinxstyleemphasis{C++ Compiler} - Toolchains required for building C++ code.

\item {} 
\sphinxstyleemphasis{Java JDK/JRE} - Specific version of the JDK/JRE that is used to build code.

\item {} 
\sphinxstyleemphasis{Gradle} - Specific version of Gradle used for building and deploying Java or C++ code

\item {} 
\sphinxstyleemphasis{WPILib Tools} - Tools used for robot enhancement

\item {} 
\sphinxstyleemphasis{WPILib Dependencies} - OpenCV, etc.

\item {} 
\sphinxstyleemphasis{VS Code Extensions} - WPILib extensions for robot code development

\end{itemize}

\begin{sphinxadmonition}{important}{Important:}
The installer creates a seperate version of VS Code for robotics development, even if VS Code is already installed locally. This is done to prevent workflows from breaking.
\end{sphinxadmonition}

This part needs to be written and tested on a apple device

This part needs to be written and tested on a linux device. Most likely will done through a VM


\section{Offline Installer}
\label{\detokenize{docs/WorldSkills/setup:offline-installer}}
Windows

macOS

Linux

\begin{sphinxadmonition}{warning}{Warning:}
Windows 7: You \sphinxstylestrong{must} install the standalone version of .NET Version 4.62+ which can be found \sphinxhref{https://support.microsoft.com/en-us/help/3151800/the-net-framework-4-6-2-offline-installer-for-windows}{here}. Before preceding!
\end{sphinxadmonition}

The offline installation files will be located in the USB provided inside the collection. Locate and run the file named \sphinxcode{\sphinxupquote{WPILibInstaller\_Windows64-2020.3.2.exe}} or \sphinxcode{\sphinxupquote{WPILibInstaller\_Windows32-2020.3.2.exe}} based on your OS.

\noindent{\hspace*{\fill}\sphinxincludegraphics{{worldskills-software-setup-1}.PNG}\hspace*{\fill}}

Installing for \sphinxstyleemphasis{All Users} will require admin privileges and install for all users on the machine.

\begin{sphinxadmonition}{note}{Note:}
Software will be installed to \sphinxcode{\sphinxupquote{C:\textbackslash{}Users\textbackslash{}Public\textbackslash{}wpilib\textbackslash{}YYYY}}. YYYY Corresponds to the currently suppored year.
\end{sphinxadmonition}

\sphinxstylestrong{Download VS Code}
\begin{quote}

Due to licensing reasons with VS Code the installer does not contain it bundled in. To overcome this hit the \sphinxstyleemphasis{Select/Download VS Code} button.

\noindent{\hspace*{\fill}\sphinxincludegraphics{{worldskills-software-setup-2}.PNG}\hspace*{\fill}}

This will open up the selector tool.

\noindent{\hspace*{\fill}\sphinxincludegraphics{{worldskills-software-setup-3}.PNG}\hspace*{\fill}}

Select the \sphinxstyleemphasis{Select Existing Download} option and then select the file \sphinxcode{\sphinxupquote{OfflineVsCodeFiles-1.41.1.zip}}. This will change back to the installer window and \sphinxstyleemphasis{Execute Install} can be run.

\noindent{\hspace*{\fill}\sphinxincludegraphics{{worldskills-software-setup-5}.PNG}\hspace*{\fill}}
\end{quote}

\sphinxstylestrong{What was just Installed}
\begin{itemize}
\item {} 
\sphinxstyleemphasis{Visual Stuido Code} - The prefered and suppored IDE for robot code development.

\item {} 
\sphinxstyleemphasis{C++ Compiler} - Toolchains required for building C++ code.

\item {} 
\sphinxstyleemphasis{Java JDK/JRE} - Specific version of the JDK/JRE that is used to build code.

\item {} 
\sphinxstyleemphasis{Gradle} - Specific version of Gradle used for building and deploying Java or C++ code

\item {} 
\sphinxstyleemphasis{WPILib Tools} - Tools used for robot enhancement

\item {} 
\sphinxstyleemphasis{WPILib Dependencies} - OpenCV, etc.

\item {} 
\sphinxstyleemphasis{VS Code Extensions} - WPILib extensions for robot code development

\end{itemize}

\begin{sphinxadmonition}{important}{Important:}
The installer creates a seperate version of VS Code for robotics development, even if VS Code is already installed locally. This is done to prevent workflows from breaking.
\end{sphinxadmonition}

This part needs to be written and tested on a apple device

This part needs to be written and tested on a linux device. Most likely will done through a VM


\chapter{World Skills Programming}
\label{\detokenize{docs/WorldSkills/programming:world-skills-programming}}\label{\detokenize{docs/WorldSkills/programming::doc}}
Programming description goes here

Java

C++

\begin{sphinxVerbatim}[commandchars=\\\{\}]
\PYG{c+c1}{//Java code input}
\PYG{n}{leftfront} \PYG{o}{=} \PYG{k}{new} \PYG{n}{Titan}\PYG{o}{(}\PYG{l+m+mi}{42}\PYG{o}{,} \PYG{l+m+mi}{0}\PYG{o}{,} \PYG{l+m+mi}{15600}\PYG{o}{)}\PYG{o}{;}
\end{sphinxVerbatim}

\begin{sphinxVerbatim}[commandchars=\\\{\}]
\PYG{c+c1}{//C++ Code inpute}
\PYG{n}{Titan} \PYG{n}{leftfront}\PYG{p}{\PYGZob{}}\PYG{l+m+mi}{42}\PYG{p}{,} \PYG{l+m+mi}{0}\PYG{p}{,} \PYG{l+m+mi}{15600}\PYG{p}{\PYGZcb{}}\PYG{p}{;}
\end{sphinxVerbatim}


\chapter{Troubleshooting}
\label{\detokenize{docs/WorldSkills/troubleshooting:troubleshooting}}\label{\detokenize{docs/WorldSkills/troubleshooting::doc}}
Page to describe World Skills troubleshooting problems


\chapter{Setting Up}
\label{\detokenize{docs/FRCTraining/setup:setting-up}}\label{\detokenize{docs/FRCTraining/setup::doc}}
Page to describe the set-up for FRC training


\chapter{FRC Programming}
\label{\detokenize{docs/FRCTraining/programming:frc-programming}}\label{\detokenize{docs/FRCTraining/programming::doc}}
Programming Description

Java

C++

\begin{sphinxVerbatim}[commandchars=\\\{\}]
\PYG{c+c1}{//Java code input}
\PYG{n}{leftFront} \PYG{o}{=} \PYG{k}{new} \PYG{n}{Titan}\PYG{o}{(}\PYG{l+m+mi}{42}\PYG{o}{,} \PYG{l+m+mi}{0}\PYG{o}{,} \PYG{l+m+mi}{15600}\PYG{o}{)}\PYG{o}{;}
\end{sphinxVerbatim}

\begin{sphinxVerbatim}[commandchars=\\\{\}]
\PYG{c+c1}{//C++ Code input}
\PYG{n}{Titan} \PYG{n}{leftFront}\PYG{p}{\PYGZob{}}\PYG{l+m+mi}{42}\PYG{p}{,} \PYG{l+m+mi}{0}\PYG{p}{,} \PYG{l+m+mi}{15600}\PYG{p}{\PYGZcb{}}\PYG{p}{;}
\end{sphinxVerbatim}


\chapter{Troubleshooting}
\label{\detokenize{docs/FRCTraining/troubleshooting:troubleshooting}}\label{\detokenize{docs/FRCTraining/troubleshooting::doc}}
Page to describe FRC troubleshooting problems


\chapter{Setting Up}
\label{\detokenize{docs/VMX/setup:setting-up}}\label{\detokenize{docs/VMX/setup::doc}}
Page to describe the set-up for the VMX


\chapter{Setting Up VMX Vision}
\label{\detokenize{docs/VMX/vision:setting-up-vmx-vision}}\label{\detokenize{docs/VMX/vision::doc}}
Page that shows how to setup VMX vision


\chapter{VMX Update}
\label{\detokenize{docs/VMX/update:vmx-update}}\label{\detokenize{docs/VMX/update::doc}}
Page that describes how to update the VMX


\chapter{Troubleshooting}
\label{\detokenize{docs/VMX/troubleshooting:troubleshooting}}\label{\detokenize{docs/VMX/troubleshooting::doc}}
Page to describe VMXtroubleshooting problems


\chapter{Titan Calibration}
\label{\detokenize{docs/Titan/calibration:titan-calibration}}\label{\detokenize{docs/Titan/calibration::doc}}
Page to describe how to calibrate Titan


\chapter{How to use Update App}
\label{\detokenize{docs/Titan/update:how-to-use-update-app}}\label{\detokenize{docs/Titan/update::doc}}
Page to describe using the update App for Titan


\chapter{Troubleshooting}
\label{\detokenize{docs/Titan/troubleshooting:troubleshooting}}\label{\detokenize{docs/Titan/troubleshooting::doc}}
Page to describe Titan troubleshooting problems


\chapter{Indices and tables}
\label{\detokenize{index:indices-and-tables}}\begin{itemize}
\item {} 
\DUrole{xref,std,std-ref}{genindex}

\item {} 
\DUrole{xref,std,std-ref}{modindex}

\item {} 
\DUrole{xref,std,std-ref}{search}

\end{itemize}



\renewcommand{\indexname}{Index}
\printindex
\end{document}